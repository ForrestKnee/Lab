% Digital Logic Report Template
% Created: 2020-01-10, John Miller

%==========================================================
%=========== Document Setup  ==============================

% Formatting defined by class file
\documentclass[11pt]{article}

% ---- Document formatting ----
\usepackage[margin=1in]{geometry}	% Narrower margins
\usepackage{booktabs}				% Nice formatting of tables
\usepackage{graphicx}				% Ability to include graphics

%\setlength\parindent{0pt}	% Do not indent first line of paragraphs 
\usepackage[parfill]{parskip}		% Line space b/w paragraphs
%	parfill option prevents last line of pgrph from being fully justified

% Parskip package adds too much space around titles, fix with this
\RequirePackage{titlesec}
\titlespacing\section{0pt}{8pt plus 4pt minus 2pt}{3pt plus 2pt minus 2pt}
\titlespacing\subsection{0pt}{4pt plus 4pt minus 2pt}{-2pt plus 2pt minus 2pt}
\titlespacing\subsubsection{0pt}{2pt plus 4pt minus 2pt}{-6pt plus 2pt minus 2pt}

% ---- Hyperlinks ----
\usepackage[colorlinks=true,urlcolor=blue]{hyperref}	% For URL's. Automatically links internal references.

% ---- Code listings ----
\usepackage{listings} 					% Nice code layout and inclusion
\usepackage[usenames,dvipsnames]{xcolor}	% Colors (needs to be defined before using colors)

% Define custom colors for listings
\definecolor{listinggray}{gray}{0.98}		% Listings background color
\definecolor{rulegray}{gray}{0.7}			% Listings rule/frame color

% Style for Verilog
\lstdefinestyle{Verilog}{
	language=Verilog,					% Verilog
	backgroundcolor=\color{listinggray},	% light gray background
	rulecolor=\color{blue}, 			% blue frame lines
	frame=tb,							% lines above & below
	linewidth=\columnwidth, 			% set line width
	basicstyle=\small\ttfamily,	% basic font style that is used for the code	
	breaklines=true, 					% allow breaking across columns/pages
	tabsize=3,							% set tab size
	commentstyle=\color{gray},	% comments in italic 
	stringstyle=\upshape,				% strings are printed in normal font
	showspaces=false,					% don't underscore spaces
}

% How to use: \Verilog[listing_options]{file}
\newcommand{\Verilog}[2][]{%
	\lstinputlisting[style=Verilog,#1]{#2}
}




%======================================================
%=========== Body  ====================================
\begin{document}

\title{ELC 2137 Lab 7: BCD}
\author{Forrest Knee}

\maketitle


\section*{Summary}

In this lab I coded the seven segment display of the Basys 3 to output numbers in binary coded decimal from the switches input.


\section*{Results}

\centering

\includegraphics[scale=.1,angle=270,trim={5cm 5cm 40cm 4cm},clip]{Lab7_Circuit}

11-Bit Double Dabble Circuit 

\includegraphics[scale=0.1,angle=270,trim={30cm 10cm 40cm 00cm},clip]{LabFirstDigit}

First Digit of 98

\includegraphics [scale=.1,angle=270,trim={24cm 00cm 47cm 5cm},clip]{LabSecondDigit}

Second Digit of 98

\includegraphics[scale=0.7]{LabAdd3_Sim_Waveform}

Add 3 Waveform

\includegraphics[scale=0.63]{Lab_BCD6_Sim_Waveform}

BCD6 Waveform

\includegraphics[scale=0.7]{Lab_BCD11_Waveform}

BCD11 Waveform

\includegraphics[scale=0.66]{LabSseg1_BCD_Sim_Waveform}

SSEG BCD Decoder Waveform

\pagebreak

\section*{Code}

\Verilog[caption = Add 3 Module]{add3.v}

\Verilog[caption = Add 3 Test]{add3_SIM.sv}

\Verilog[caption = BCD 6 Module]{bcd6.sv}

\Verilog[caption = BCD 6 Test]{BCD6_SIM.sv}

\Verilog[caption = BCD 11 Module]{bcd11.sv}

\Verilog[caption = BCD 11 Test]{BCD11_SIM.sv}

\Verilog[caption = Seven Segment Decoder to BCD]{sseg1_BCD.sv}

\Verilog[caption = Decoder Test]{sseg1bcdSim.sv}



\end{document}
