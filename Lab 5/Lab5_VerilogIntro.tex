% Digital Logic Report Template
% Created: 2020-01-10, John Miller

%==========================================================
%=========== Document Setup  ==============================

% Formatting defined by class file
\documentclass[11pt]{article}

% ---- Document formatting ----
\usepackage[margin=1in]{geometry}	% Narrower margins
\usepackage{booktabs}				% Nice formatting of tables
\usepackage{graphicx}				% Ability to include graphics

%\setlength\parindent{0pt}	% Do not indent first line of paragraphs 
\usepackage[parfill]{parskip}		% Line space b/w paragraphs
%	parfill option prevents last line of pgrph from being fully justified

% Parskip package adds too much space around titles, fix with this
\RequirePackage{titlesec}
\titlespacing\section{0pt}{8pt plus 4pt minus 2pt}{3pt plus 2pt minus 2pt}
\titlespacing\subsection{0pt}{4pt plus 4pt minus 2pt}{-2pt plus 2pt minus 2pt}
\titlespacing\subsubsection{0pt}{2pt plus 4pt minus 2pt}{-6pt plus 2pt minus 2pt}

% ---- Hyperlinks ----
\usepackage[colorlinks=true,urlcolor=blue]{hyperref}	% For URL's. Automatically links internal references.

% ---- Code listings ----
\usepackage{listings} 					% Nice code layout and inclusion
\usepackage[usenames,dvipsnames]{xcolor}	% Colors (needs to be defined before using colors)

% Define custom colors for listings
\definecolor{listinggray}{gray}{0.98}		% Listings background color
\definecolor{rulegray}{gray}{0.7}			% Listings rule/frame color

% Style for Verilog
\lstdefinestyle{Verilog}{
	language=Verilog,					% Verilog
	backgroundcolor=\color{listinggray},	% light gray background
	rulecolor=\color{blue}, 			% blue frame lines
	frame=tb,							% lines above & below
	linewidth=\columnwidth, 			% set line width
	basicstyle=\small\ttfamily,	% basic font style that is used for the code	
	breaklines=true, 					% allow breaking across columns/pages
	tabsize=3,							% set tab size
	commentstyle=\color{gray},	% comments in italic 
	stringstyle=\upshape,				% strings are printed in normal font
	showspaces=false,					% don't underscore spaces
}

% How to use: \Verilog[listing_options]{file}
\newcommand{\Verilog}[2][]{%
	\lstinputlisting[style=Verilog,#1]{#2}
}




%======================================================
%=========== Body  ====================================
\begin{document}

\title{ELC 2137 Lab 5: Intro to Verilog}
\author{Forrest Knee}

\maketitle


\section*{Summary}

This lab involved creating a half adder, full adder, and 2-bit adder/subtractor and simulating them in verilog.


\section*{Q\&A}

4. Do the simulations match the expected output values?

	Yes, my simulations do perform as expected.

5. What is one thing that you still don't understand about verilog?
	
	There are many many options that we were told to skip through and select certain settings when creating our project and file. I did not understand many of those choices.

\section*{Results}
\begin{figure}
\includegraphics[scale=0.7,trim=15cm 16cm .5cm 4cm,clip]{haScreenShot}
\caption{Half Adder}
\end{figure}

\begin{figure}
\includegraphics[scale=0.2,trim=50cm 50cm 1cm 13cm,clip]{faScreenShot}
\caption{Full Adder}
\end{figure}

\begin{figure}
\includegraphics[scale=0.32,trim=32cm 30cm 0.5cm 7cm,clip]{AddSubbScreenShot}
\caption{2 Bit Adder Subtractor}
\end{figure}

\begin{figure}
\includegraphics[scale=0.5]{lab5_block}
\end{figure}

\clearpage


\section*{Code}


\Verilog[caption=Half Adder]{halfadder.v}

\Verilog[caption=Half Adder Test]{halfadder_SIM.v}

\Verilog[caption=Full Adder]{fulladder.v}

\Verilog[caption=Full Adder Test]{fulladder_SIM.v}

\Verilog[caption=2 Bit Adder Subtractor]{addersubtractor.v}

\Verilog[caption=2 Bit Adder Subtractor Test]{addsub_SIM.v}

\end{document}
